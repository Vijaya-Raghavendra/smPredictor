\documentclass{article}
\usepackage{graphicx}
\usepackage{fancyhdr}
\usepackage{float}
\usepackage[hidelinks]{hyperref}
\usepackage{letterspace}
\usepackage{xcolor}
\usepackage{pagecolor}
\usepackage[a4paper, hmargin=1in, bottom=1in]{geometry}
\usepackage{fontspec}
\usepackage{amsmath}
\usepackage{eso-pic}
\usepackage{tikz}
\usepackage{dirtree}
\usepackage{enumitem}
\usepackage{listings}
\usepackage{xcolor}

\usetikzlibrary{trees}
\tikzstyle{every node}=[draw=gray, very thick, anchor=west]

\lstdefinelanguage{bash}{
    sensitive=true,
    morekeywords={mkdir, python3, source, pip, python},
    morecomment=[l]{\#},
    morestring=[b]",
}

\lstset{
    language=bash,
    basicstyle=\ttfamily\small,
    keywordstyle=\color{blue},
    commentstyle=\color{gray},
    stringstyle=\color{orange},
    showstringspaces=false,
    breaklines=true,
    frame=single,
    columns=fullflexible,
}

\usetikzlibrary{shapes.geometric, arrows}
\setlist[itemize]{label=\Large$\bullet$}
\tikzstyle{startstop} = [rectangle, rounded corners, minimum width=3cm, minimum
height=1cm, text centered, draw=black, fill=red!30]
\tikzstyle{process} = [rectangle, minimum width=3cm, minimum height=1cm, text
centered, draw=black, fill=orange!30]
\tikzstyle{arrow} = [thick,->,>=stealth]

\setmainfont{Rajdhani}[BoldFont = Rajdhani Semibold]
\definecolor{grey}{RGB}{40, 40, 40}
\definecolor{section_color}{RGB}{0, 45, 98}
\definecolor{subsection_color}{RGB}{0, 48, 143}
\definecolor{subsubsection_color}{RGB}{0, 0, 255}
\definecolor{text_color}{RGB}{50, 50, 50}
\definecolor{bgcolor}{RGB}{240, 240, 240}

\begin{document}

\begin{titlepage}

    \pagecolor{bgcolor}
    \begin{center}

        \includegraphics[width=0.7\textwidth]{/home/mradul/Pictures/globals/iitb_logo.png}

        \vspace{100pt}

        \makebox[\textwidth][c]{%
            {\fontfamily{lmdh}\selectfont \fontsize{35}{45}\selectfont
                    \textcolor{section_color}{\textls[40]{Project Report}}}
        }
        \
        \vspace{10pt}

        \makebox[\textwidth][c]{{\selectfont \fontsize{17}{10}\selectfont
                    \textcolor{grey}{{Artificial Intelligence and Machine
                                Learning}}}}

        \vfill

        {\fontsize{16}{30}\selectfont {\textcolor{grey}{Mradul Sonkar |
                    Vijaya Raghavendra}}}\\
        \vspace{5pt}
        {\fontsize{16}{30}\selectfont {\textcolor{grey}{Instructor:
                    Pushpak Bhattacharyya
                }}}

    \end{center}

\end{titlepage}

\nopagecolor\

\section*{\fontsize{20}{20}\selectfont Idea:}

 {\noindent \fontsize{12}{16}\selectfont We aim to develop a \textbf{real-time
      stock market
      price predictor} using an online learning approach. The core goal is to
  build a
  model that, given the past few intervals of stock data, can \textbf{predict
      the
      closing price of the next interval} and continuously \textbf{update
      itself with
      new data} as it arrives.

  Traditional time series models rely on batch learning and require
  retraining
  when new data is available. In contrast, our approach leverages
  \textbf{incremental learning} using the \texttt{deep\_river} library,
  which
  allows the model to \textbf{learn and predict in a streaming fashion},
  making
  it suitable for dynamic financial environments.

  We use:
  \begin{itemize}
      \item A custom \textbf{LSTM model built in PyTorch}, trained to learn
            temporal dependencies in stock price movement.
      \item The LSTM is wrapped inside \textbf{\texttt{deep\_river}'s\space
                \texttt{RollingRegressor}}, enabling online training.
      \item The model takes input features such as \texttt{Open},
            \texttt{Close},
            \texttt{High}, \texttt{Low}, and \texttt{Volume}, along with
            their lagged
            values (e.g., Open$_{t-1}$, Volume$_{t-2}$).
      \item For each time step, it predicts the next closing price and
            updates
            its weights using the true observed price.
  \end{itemize}

  The stock data is fetched dynamically using the \texttt{yfinance} API,
  allowing the model to operate on real-time or historical data streams.

 }

\section*{\fontsize{20}{20}\selectfont Structure:}

{\fontsize{14}{20} \selectfont The source code of the project can be found
\textbf{\href{https://drive.google.com/file/d/1W9iylU5aluezGqYUcOf1AgIQyAChNvWV}{here}.} Follow the following instructions to setup the project.
\dirtree{%
    .1 ..
    .2 main.ipynb.
    .2 main.py.
    .2 README.md.
    .2 requirements.txt.
    .2 data/.
    .3 raw/.
    .3 processed/.
    .2 src/.
    .3 dataHandler.py.
    .3 lstmModule.py.
    .3 model.py.
    .3 processing.py.
    .3 tickers.json.
}
}
\newpage

\begin{itemize}
    \item \texttt{main.ipynb}: Jupyter notebook version of main.py.
    \item \texttt{main.py}: Starting point of the project.
    \item \texttt{README.md}: Instructions to use the project.
    \item \texttt{src/}: Source folder containing all files.
    \item \texttt{tickers.json}: JSON file containing all the tickers for which
          we are fetching the data.
\end{itemize}

\vspace{8.0pt}

\section*{\fontsize{20}{20}\selectfont Usage:}
\begin{itemize}
    \item Start by creating a virtual environment, name it \textbf{project},
          and activate it.
          \begin{lstlisting}[language=bash]
            python3 -m venv project
            source project/bin/activate
            pip install -r requirements.txt
        \end{lstlisting}
    \item Create the folders to fetch the data in, name it \textbf{data}.
          \begin{lstlisting}[language=bash]
                mkdir data
                mkdir data/raw
                mkdir data/processed
            \end{lstlisting}
    \item Run the file \textbf{main.py} or optionally \textbf{main.ipynb} for
          more analysis.
          \begin{lstlisting}
                python main.py
            \end{lstlisting}

\end{itemize}

\section*{\fontsize{20}{20}\selectfont Current Progress and Future
  PoA}

 {\noindent \fontsize{12}{16}\selectfont
  So far, we have developed a model to predict future stock prices using an
  LSTM-based architecture wrapped in a rolling regressor. The model takes
  historical stock data as input and outputs a predicted price for the upcoming
  interval.
  \\[0.3cm]
  \noindent Future improvements include:
  \begin{itemize}
      \item Fetching current financial news from online sources and analyzing
            its
            impact on price movements.
      \item Implementing functionality to fetch real-time stock prices at
            regular
            intervals (e.g., every minute).
      \item Developing a simple user interface to visualize predicted and
            actual
            price movements on graphs. This interface could be built using
            Flask
            or a
            similar lightweight web framework.
  \end{itemize}}

\end{document}